\chapter{C++ Template}
\label{chap:cpp-template}

\section{Basic Template}

This is the basic template that anyone can use to write the code in c++.

\begin{lstlisting}[language=C++, caption={Main C++ Template}]
#include <bits/stdc++.h>

using namespace std;

void solve(){
    //Code Here....
}
void init_code(){
    ios_base :: sync_with_stdio(false);
    cin.tie(nullptr);
    cout.tie(nullptr);
}
int main(){
    init_code();
    solve();
    return 0;
}

\end{lstlisting}
\newpage
\section{Debugging Template}
\begin{lstlisting}[language=C++, caption={Debugging Template}]
template < class c > struct rge {
    c b, e;
};
template < class c > rge<c> range(c i, c j){
    return rge<c>{i, j};
}
template < class c > auto dud(c* x) -> decltype(cerr << *x, 0);
template < class c > char dud(...);

struct debug {
    ~debug() { cerr << endl; }
    template < class c > typename enable_if<sizeof dud<c>(0) != 1, debug&>::type operator<<(c i) {
        cerr << boolalpha << i;
        return * this;
    }
    template < class c > typename enable_if<sizeof dud<c>(0) == 1, debug&>::type operator<<(c i) {
        return * this << range(begin(i), end(i)); 
    }
    template < class c, class b > debug & operator << (pair < b, c > d) {
        return * this << "(" << d.first << ", " << d.second << ")";
    }
    template < class c > debug & operator <<(rge<c> d) {
        *this << "[";
        for (auto it = d.b; it != d.e; ++it)
            *this << ", " + 2 * (it == d.b) << *it;
        return * this << "]";
    }
};  
#define imie(...) " [" << #__VA_ARGS__ " : " << (__VA_ARGS__) << "]"
\end{lstlisting}
\newpage
\section{Combined Template}

Here is a combined template for development that includes both the main structure and debugging utilities.

\begin{lstlisting}[language=C++, caption={Combined Template}]
/*
*
*    Author : Girish Kumar Goyal.
*
*/

#include <bits/stdc++.h>

using namespace std;

template < class c > struct rge {
    c b, e;
};
template < class c > rge<c> range(c i, c j){
    return rge<c>{i, j};
}
template < class c > auto dud(c* x) -> decltype(cerr << *x, 0);
template < class c > char dud(...);
struct debug {
    ~debug() { cerr << endl; }
    template < class c > typename enable_if<sizeof dud<c>(0) != 1, debug&>::type operator<<(c i) {
        cerr << boolalpha << i;
        return * this;
    }
    template < class c > typename enable_if<sizeof dud<c>(0) == 1, debug&>::type operator<<(c i) {
        return * this << range(begin(i), end(i)); 
    }
    template < class c, class b > debug & operator << (pair < b, c > d) {
        return * this << "(" << d.first << ", " << d.second << ")";
    }
    template < class c > debug & operator <<(rge<c> d) {
        *this << "[";
        for (auto it = d.b; it != d.e; ++it)
            *this << ", " + 2 * (it == d.b) << *it;
        return * this << "]";
    }
};  
#define imie(...) " [" << #__VA_ARGS__ " : " << (__VA_ARGS__) << "]"

void solve(){
    //Code Here....
}
void init_code(){
    ios_base :: sync_with_stdio(false);
    cin.tie(nullptr);
    cout.tie(nullptr);
}
int main(){
    init_code();
    solve();
    return 0;
}
\end{lstlisting}
\section{Using the Debugging Template}

To use the debugging template, you can print variables or expressions during development by using the `debug()` object in combination with the `imie(...)` macro. This will output both the name of the variable and its value, which is extremely useful for tracing bugs.

\textbf{Syntax:}
\begin{lstlisting}[language=C++]
debug() << imie(variable1) << imie(variable2) << imie(expression);
\end{lstlisting}

\textbf{Example:}

Here's a complete example demonstrating the use of `debug()` and `imie(...)`:

\begin{lstlisting}[language=C++, caption={Debugging Example}]
void solve() {
    int a = 42;
    string s = "hello";
    vector<int> v = {1, 2, 3};
    
    debug() << imie(a) << imie(s) << imie(v);
}

void init_code(){
    ios_base :: sync_with_stdio(false);
    cin.tie(nullptr);
    cout.tie(nullptr);
}

int main(){
    init_code();
    solve();
    return 0;
}
\end{lstlisting}


\section{C++ program time calculation}

\textbf{Example:}

\begin{lstlisting}[language=C++, caption={CPP Code time Example}]
#include <bits/stdc++.h>

using namespace std;

void solve() {
    // Code here....
}

void init_code(){
    ios_base :: sync_with_stdio(false);
    cin.tie(nullptr);
    cout.tie(nullptr);
}

int main(){
    init_code();
    clock_t start, end;
    double cpu_time;
    start = clock();
    solve();
    cpu_time = ((double)(end - start)) / CLOCKS_PER_SEC;
    cout << cpu_time << "\n";
    return 0;
}
\end{lstlisting}

\newpage

\section{Test case generator program in C++}

\textbf{Example:}

Test case generator in c++ basically the code to generate random testcases like this.
\begin{verbatim}
2
5
1 2 3 4 5
4
2 3 1 43
\end{verbatim}

\begin{lstlisting}[language=C++, caption={Test case generator}]
#include<bits/stdc++.h>

using namespace std;

int naxTest = 20;
int naxArraySize = 50;
int naxValue = 5000;

vector<int> generateTestCase(int size, int maxValue) {
    vector<int> testCase(size);
    for (int i = 0; i < size; i++) {
        testCase[i] = rand() % maxValue + 1;
    }
    return testCase;
}

int main() {
    srand(time(0));
    int numTestCases = rand() % naxTest + 1;
    int maxArraySize = rand() % naxArraySize + 1;
    int maxValue = rand() % naxValue + 1;
    cout << numTestCases << "\n";
    for (int i = 0; i < numTestCases; i++) {
        int arraySize = rand() % maxArraySize + 1;
        cout << arraySize << "\n";
        vector<int> testCase = generateTestCase(arraySize, maxValue);
        // sort(testCase.begin(), testCase.end());
        for (int num : testCase) {
            cout << num << " ";
        }
        cout << "\n";
    }
    return 0;
}
\end{lstlisting}

\section{Script for stress testing of c++ code}

\textbf{Syntax:}
\begin{lstlisting}[language=C++]
./stress.sh
\end{lstlisting}

\textbf{Example:}

The following script automates compilation and stress testing of a C++ solution against generated test cases:

\begin{lstlisting}[style=bashstyle]
#!/bin/bash

black=$(tput setaf 0)
red=$(tput setaf 1)
green=$(tput setaf 2)
yellow=$(tput setaf 3)
blue=$(tput setaf 4)
magenta=$(tput setaf 5)
cyan=$(tput setaf 6)
white=$(tput setaf 7)
bold=$(tput bold)
reset=$(tput sgr0)
CPP_VERSION="c++17"
COMPILE_FLAGS="-Wall -Wextra -O2"
TEST_GEN_FILE="./stress_tester/test_gen.cpp"
MAIN_FILE="sol.cpp"
INPUT_FILE="in"
MAX_TESTS=5
TIME_LIMIT=2  # seconds per test case

usage() {
    echo -e "${bold}${cyan}Usage:${reset} $(basename "$0") [-t <num_tests>]"
    exit 1
}
while getopts "t:h" opt; do
    case $opt in
        t)
            MAX_TESTS="$OPTARG"
            ;;
        h)
            usage
            ;;
        *)
            usage
            ;;
    esac
done

log_info() {
    echo -e "${bold}${blue}[INFO]${reset} $1"
}
log_success() {
    echo -e "${bold}${green}[SUCCESS]${reset} $1"
}
log_error() {
    echo -e "${bold}${red}[ERROR]${reset} $1"
}
log_warning() {
    echo -e "${bold}${yellow}[WARNING]${reset} $1"
}
check_files() {
    echo ""
    echo "---------------------------------------------------------"
    local missing=0
    if [ ! -f "$MAIN_FILE" ]; then
        log_error "Main solution file ${yellow}$MAIN_FILE${reset} not found!"
        missing=1
    fi
    if [ ! -f "$TEST_GEN_FILE" ]; then
        log_error "Test case generator file ${yellow}$TEST_GEN_FILE${reset} not found!"
        missing=1
    fi
    if [ $missing -eq 1 ]; then
        exit 1
    fi
}
compile_file() {
    local src_file="$1"
    local exe_file="$2"
    local extra_flags="$3"
    log_info "Compiling ${yellow}$src_file${reset}..."
    local start_ns=$(date +%s%N)
    compile_output=$(g++ -std="$CPP_VERSION" $COMPILE_FLAGS $extra_flags "$src_file" -o "$exe_file" 2>&1)
    if [ $? -ne 0 ]; then
        log_error "Compilation failed for ${yellow}$src_file${reset}."
        echo -e "${red}Compiler output:${reset}"
        echo "$compile_output"
        exit 1
    fi
    local end_ns=$(date +%s%N)
    local compile_time_ns=$((end_ns - start_ns))
    local compile_time_ms=$(echo "scale=3; $compile_time_ns/1000000" | bc)
    log_success "Compiled ${yellow}$src_file${reset} to ${magenta}$exe_file${reset} in ${cyan}${compile_time_ms} ms${reset}."
}
stress_test() {
    local accepted=0
    local failed=0
    local total_test_time_ns=0
    echo -e "\n${bold}${red}Starting stress testing with ${yellow}$MAX_TESTS${red} test cases...${reset}"
    for (( i=1; i<=MAX_TESTS; i++ )); do
        echo -e "\n${bold}${blue}======= Test case #$i =======${reset}"
        ./generator > "$INPUT_FILE"
        echo -e "${bold}${magenta}[Input]:${reset}"
        cat "$INPUT_FILE"
        echo ""
        local start_ns=$(date +%s%N)
        output=$(timeout $TIME_LIMIT ./sol < "$INPUT_FILE" 2>&1)
        local ret_code=$?
        local end_ns=$(date +%s%N)
        local elapsed_ns=$((end_ns - start_ns))
        total_test_time_ns=$((total_test_time_ns + elapsed_ns))
        local elapsed_ms=$(echo "scale=3; $elapsed_ns/1000000" | bc)
        echo -e "${bold}${cyan}[Output]:${reset}"
        echo -e "$output"
        if [ $ret_code -eq 0 ]; then
            echo -e "${bold}${green}Test case #$i: Accepted in ${blue}${elapsed_ms} ms${reset}"
            ((accepted++))
        elif [ $ret_code -eq 124 ]; then
            echo -e "${bold}${red}Test case #$i: Time Limit Exceeded (TLE) after ${yellow}${elapsed_ms} ms${reset}"
            ((failed++))
        elif [ $ret_code -eq 137 ]; then
            echo -e "${bold}${red}Test case #$i: Memory Limit Exceeded (MLE) in ${yellow}${elapsed_ms} ms${reset}"
            ((failed++))
        elif [ $ret_code -eq 139 ]; then
            echo -e "${bold}${red}Test case #$i: Runtime Error (Segmentation Fault) in ${yellow}${elapsed_ms} ms${reset}"
            ((failed++))
        else
            echo -e "${bold}${red}Test case #$i: Runtime Error (exit code $ret_code) in ${yellow}${elapsed_ms} ms${reset}"
            ((failed++))
        fi
        echo -e "${bold}${white}=====================================${reset}"
    done
    total_test_time_ms=$(echo "scale=3; $total_test_time_ns/1000000" | bc)
    avg_time=$(echo "scale=3; $total_test_time_ms/$MAX_TESTS" | bc)
    echo -e "\n${bold}${red}Stress testing completed.${reset}"
    echo -e "${bold}${green}[Accepted]: $accepted${reset}    ${bold}${red}[Failed]: $failed${reset}"
    echo -e "${bold}${cyan}[Total testing time]: ${total_test_time_ms} ms${reset}"
    echo -e "${bold}${magenta}[Average time per test]: ${avg_time} ms${reset}"
    echo ""
}
check_files
compile_file "$TEST_GEN_FILE" "generator" ""
compile_file "$MAIN_FILE" "sol" ""
stress_test
\end{lstlisting}

