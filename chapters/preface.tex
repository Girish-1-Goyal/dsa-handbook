\chapter*{\color{black}Preface}
\addcontentsline{toc}{chapter}{Preface}

\vspace{1em}

\begin{spacing}{1.2}
\color{black}

The purpose of this book is to give you a thorough introduction to \textbf{data structures}. It is assumed that you already have a basic understanding of programming, but no previous background in data structures or algorithms is needed.

\vspace{1em}

This book is especially intended for \textit{students} who want to learn how to efficiently organize and process data, and to build a strong foundation in algorithmic problem solving. However, it is equally suitable for professionals looking to strengthen their technical fundamentals or prepare for coding interviews.

\vspace{1em}

Each chapter is designed with conceptual clarity, illustrative C++ code. It includes fundamental structures such as arrays and linked lists, dive into advanced trees and graphs, and discuss both classic and modern algorithms with a focus on time and space complexity.

\vspace{1em}

I believe that theory must meet practice. Therefore, this book contains visual diagrams, sample programs.

\vspace{1em}

I would like to express my gratitude to my mentors, peers, and the open-source community that continues to shape and evolve computer science education. Their contributions, directly or indirectly, have inspired many parts of this book.

\vspace{1em}

This book is more than a technical reference — it is a roadmap for anyone who wishes to unlock the power of structured thinking and logical problem solving.

\vspace{1em}

\textit{To the readers — approach each chapter with curiosity, code with confidence, and always keep learning.}

\end{spacing}

\vspace{3em}

\begin{flushright}
\textcolor{gray!60!black}{\itshape India, April 2025} \\
\textbf{\textcolor{black}{Girish Kumar Goyal}}
\end{flushright}
